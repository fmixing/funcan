\documentclass{article}

\usepackage[sumlimits,intlimits]{amsmath}
\usepackage{enumitem,linegoal}
\usepackage{calc}
\usepackage{wrapfig}
\usepackage{blindtext}
\usepackage{tikz}
\usepackage{yhmath}
\usepackage{amssymb}
\usepackage{amsthm}
\usepackage{amsmath}
\usepackage{mathtext}
\usepackage[T1,T2A]{fontenc}
\usepackage[utf8]{inputenc}
\usepackage[russian]{babel}
%\usepackage{geometry}
\usepackage[left=2cm,right=2cm,top=2cm,bottom=2cm,bindingoffset=0cm]{geometry}
\usepackage[mathscr]{euscript}
\usepackage{microtype}
\usepackage{enumitem}
\usepackage{bm}
\usepackage{listings}
\usepackage{cancel}
\usepackage{proof}
\usepackage{epigraph}
\usepackage{titlesec}
\usepackage{mathtools}
%\setmainfont[Ligatures=TeX,SmallCapsFont={Times New Roman}]{Palatino Linotype}
\DeclareMathOperator{\grad}{grad}
\DeclareMathOperator{\Div}{div}
\DeclareMathOperator{\rot}{rot}
\DeclareMathOperator{\im}{Im}
\DeclareMathOperator{\Ker}{Ker}
\DeclareMathOperator{\codim}{codim}
\DeclareMathOperator{\Cl}{Cl}
\DeclareMathOperator{\Int}{Int}
\selectlanguage{russian}


\title{%
	Функциональный анализ 6 семестр \\
	\large Конспект лекций Додонова Н. Ю.}
\author{shared with $\heartsuit$ by artemZholus}
\date{}

\begin{document}

\theoremstyle{definition}
\newtheorem*{definition}{Определение}
\theoremstyle{plain}
\newtheorem{theorem}{Теорема}[section]
\newtheorem{axiom}{Аксиома}
\newtheorem{lemma}[theorem]{Лемма}
\newtheorem{statement}[theorem]{Утверждение}
\newtheorem{nb}[theorem]{N. B.}
\newtheorem{corollary}[theorem]{Следствие}
\theoremstyle{remark}
\newtheorem*{example}{Пример}
\newtheorem{property}[theorem]{Свойство}

\newcommand{\todo}{\textsc{\textbf{TODO}}}
\newcommand{\abs}[1]{\left|#1\right|}
\newcommand{\norm}[1]{\left\|#1\right\|}
\newcommand{\normp}[1]{\norm{#1}_p}
\newcommand{\normpp}[2]{\norm{#1}_{#2}}
\newcommand{\intl}[1]{\int\limits_{#1}}
\newcommand{\veel}[1]{\bigvee\limits_{#1}}
\newcommand{\veelr}[2]{\bigvee\limits_{#1}^{#2}}
\newcommand{\defeq}{\mathrel{\stackrel{\makebox[0pt]{\mbox{\normalfont\tiny def}}}{=}}}
\makeatletter
\newcommand*{\rom}[1]{\expandafter\@slowromancap\romannumeral #1@}
\makeatother
\newcommand{\iintl}[1]{\iint\limits_{#1}}
\newcommand{\pdiff}[2]{\frac{\partial #1}{\partial #2}}
\newcommand{\intlr}[2]{\int\limits_{#1}^{#2}}
\newcommand{\suml}[1]{\sum\limits_{#1}}
\newcommand{\sumlr}[2]{\sum\limits_{#1}^{#2}}
\newcommand{\feps}{\forall\varepsilon}
\newcommand{\Epsilon}{\varepsilon}
\newcommand{\scalarp}[2]{\langle #1 , #2\rangle}
\newcommand{\set}[1]{\left\{#1\right\}}
\renewcommand{\phi}{\varphi}
\maketitle
\tableofcontents
\newpage
\section{Сопряженный оператор}

Здесь и далее, если не оговорено иного, считаем, что мы находимся в $В$-пространствах.

\begin{definition}[Сопряженное пространство]
	$X^* = \left\{ f : X \xrightarrow[\text{непр.}]{\text{лин.}} \mathbb{R} \right\}$ - пространство сопряженное к $X$.
\end{definition}
Заметим, что это пространство линейных функционалов, а значит, мы можем ввести в нем норму, как норму линейного функционала.
\begin{gather}
	\norm{f} = \sup\limits_{\norm{x} \leqslant 1} \abs{f(x)} 
\end{gather}
По свойствам числовой оси, получаем, что $X^*$ - всегда банахово (независимо от $X$).

Рассмотрим теперь $A \in \mathcal{L}(X, Y)$. Пусть $f(x) = \phi(Ax)$, где $\phi \in Y^*$.
\begin{definition}
	Сопряженный оператор к $A$ имеет вид: $A^*(\phi) = \phi \circ A$.
\end{definition}
\begin{statement}
	Если $A$ - нерперывный, то $A^*$ - тоже непрерывный.
\end{statement}
\begin{proof}
	Пусть $A$ - непрерывен, тогда он ограничен. Тогда мы можем написать
	\begin{gather}
		\norm{A^*(\phi)} \leqslant \norm{\phi} \cdot \norm{A} \label{1.2}
	\end{gather}
	Переходя к $\sup$ по $\phi$, получаем, что нужно.
\end{proof}
\begin{theorem}
	$\norm{A^*} = \norm{A}$ \label{th1.2}
\end{theorem}
\begin{proof}
	Мы доказали неравенство в одну сторону (неравенство \ref{1.2}). Докажем в другую. 
	По определению $\sup$, $\forall \Epsilon > 0, \exists x_{\Epsilon} : \norm{x_{\Epsilon}} = 1 \Rightarrow \norm{A} - \Epsilon < \norm{Ax_{\Epsilon}}$.
	Пусть $Z = \mathcal{L}(Ax_{\Epsilon})$. Рассмотрим $f : Z \rightarrow \mathbb{R}$, $f(z) = \alpha\norm{Ax_{\Epsilon}}$. Очевидно, что $f \in Y^*$. 
	Поэтому, по теореме Хана-Банаха, распростваним $f$ на все $Y$, и назовем ее $\phi_{\Epsilon}$.
	Тогда, по свойствам $f$, $\norm{\phi_{\Epsilon}} = 1, \phi_{\Epsilon}(Ax_{\Epsilon}) = \norm{Ax_{\Epsilon}}$.
	Слудовательно, $\norm{A} - \Epsilon < \phi_{\Epsilon}(Ax_{\Epsilon}) = A^*(\phi_{\Epsilon}, x_{\Epsilon})$.
	Тогда, $\norm{A} - \Epsilon < \norm{A^*} \cdot \norm{\phi_{\Epsilon}} \cdot \norm{x_{\Epsilon}} = \norm{A^*}$. 
	Переходя к $\sup$ по $\Epsilon$ получаем нужное неравенство.
\end{proof}

Пример: \todo

\begin{theorem}[Теорема Рисса]
	Пусть $H$ - гильбертово пространство. Тогда $\forall f \in H^*$, $f$ можно представить как $f(x) = \scalarp{x}{y}$, где $y \in H, \norm{f} = \norm{y}$.
\end{theorem}
\begin{proof}
	Докажем в 3 этапа:
	\begin{enumerate}
		\item Построим соотвестсвующий функционал по данному $y$.
		\item Докажем, что этому функционалу соответствует только один $y$.
		\item Найдем $y$ для данного функционала $f$.
	\end{enumerate}
	\begin{enumerate}[leftmargin=*]
	\item
	Пусть $g(x) = \scalarp{x}{y}$. Очевидно, что это линейный функционал. 
	По неравенству Шварца $\abs{g(x)} \leqslant \norm{y} \norm{x} \Rightarrow \norm{g} \leqslant \norm{y}$. Это значит, что $g$ - ограничен.
	Возьмем $x = \frac{y}{\norm{y}}$. 
	\begin{gather*}
		g(\frac{y}{\norm{y}}) = \scalarp{\frac{y}{\norm{y}}}{y} = \frac{1}{\norm{y}}\scalarp{y}{y} = \norm{y}
	\end{gather*}
	Сопоставляя это с тем, что $\norm{g} \leqslant \norm{y}$, получаем, что $\norm{g} = \norm{y}$.
	\item
	Пусть для какого-то $\widetilde{y}$, $g(x) = \scalarp{x}{\widetilde{y}}$.
	Тогда $0 = \scalarp{x}{y} - \scalarp{x}{\widetilde{y}} = \scalarp{x}{y - \widetilde{y}}$. Пусть $x = y - \widetilde{y}$,  
	$\Rightarrow \scalarp{y - \widetilde{y}}{y - \widetilde{y}} = 0 \Rightarrow y = \widetilde{y}$
	\item
	Рассмотрим произвольный функционал $f \in H^*$. Как известно, $\Ker f$ - гиперплоскость, т.е. 
	$\codim H_1 = \dim H_2 = 1$, где $H_1 = \Ker f, H_2 = H_1^\bot$, и $H = H_1 \oplus H_2$.
	Это по определению значит, что $x$ единственным образом представим как $x = x_1 + x_2$, где $x_1 \in H_1, x_2 \in H_2$.
	Поэтому, $f(x) = f(x_1) + f(x_2) = f(x_2) = f(\alpha e) = \alpha \cdot f(e)$, так как $x_1 \in \Ker f$ а $e$ - базисный вектор из $H_2$.
	Итак, $\alpha \cdot f(x) = \scalarp{x}{y} \Leftrightarrow f(e) = \scalarp{e}{y}$. 
	Очевидно, $y$ можно брать из $H_2$, так как если у него будет компонента из $\Ker f$, то она будет ортогональна $e$. 
	Поэтому, считаем, что $y = \beta e$.
	Получаем $f(e) = \scalarp{e}{\beta e} = \beta \cdot \norm{e}^2$. Положим $\beta = \frac{f(e)}{\norm{e}^2}$, тогда $y = \frac{f(e)}{\norm{e}^2}e$.
	\end{enumerate}
\end{proof}


Пример: \todo

\section{Ортогональное дополнение в банаховых пространствах}


\begin{definition}[Ортогональное дополнение в $B$- пространстве]
	Пусть $S \subset X$.

	 Тогда $S^\bot = \left\{ f \mid f \in X^*, \forall x \in S \Rightarrow f(x) = 0 \right\}$.
\end{definition}

\begin{definition}[Ортогональное дополнение в сопряженном пространстве]
	Пусть $S \subset X^*$.

	 Тогда $S^\bot = \left\{ x \mid x \in X, \forall f \in S \Rightarrow f(x) = 0 \right\}$.
\end{definition}
Заметим, что независимо от $S$, $S^\bot$ - замкнуто, в силу непрерывности $f(x)$

\begin{statement}
	\begin{minipage}[t]{\linegoal}
	\begin{enumerate}[leftmargin=*]
		\item 	$X^\bot = \left\{ 0\right\}$
		\item   $X^{*\bot} = \left\{ 0\right\}$
	\end{enumerate}
	\end{minipage}
\end{statement}
\begin{proof}
	\begin{enumerate}[leftmargin=*]
		\item $f \in X^\bot$, Если $\forall x \in X, f(x) = 0 $, то $f \equiv 0$
		\item Рассмотрим $\forall f \in X^*$, очевидно, $f(0) = 0$, а это значит, что
		$0 \in X^{*\bot}$. Предположим, $\exists x_0 \neq 0 : x_0 \in X^{*\bot}$.
		По теореме Хана-Банаха (а точнее, по следствию, которое мы доказали в теореме \ref{th1.2}), 
		$\exists f \in X^*$, такой, что
		$f(x_0) = \norm{x_0} \neq 0$, следовательно, $x_0 \not\in X^{*\bot}$.
	\end{enumerate}
\end{proof}

\begin{definition}
	$R(A) \defeq \left\{ Ax \mid x \in X\right\}$
\end{definition}

\begin{theorem}
	$\Cl R(A) = \left( \Ker A^*\right)^\bot$
\end{theorem}
\begin{proof}
	\todo
\end{proof}
\begin{theorem}
	$R(A) = \Cl R(A) \Rightarrow \Cl R(A^*) = \left( \Ker A\right)^\bot$
\end{theorem}
\begin{proof}
	\todo
\end{proof}

\end{document}
