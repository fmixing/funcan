\section{Элементы спектральной теории линейных операторов}

\subsection{Определение спектра и резольвенты оператора}

\begin{definition}[регулярная точка]
	Число $\lambda \in \mathbb{C}$, называется \textit{регулярной точкой}
	для оператора $A$, если оператор $\lambda I - A$ "--- непрерывно обратим.
\end{definition}

\begin{definition}[резольвента]
	Множество всех регулярных точек называется \textit{резольвентой} (обозначается $\rho(A)$)
	оператора $A$.
\end{definition}

\begin{definition}[резольвентный оператор]
	Оператор $R_\lambda(A) = \left( \lambda I - A\right)^{-1}$ называется 
	\textit{резольвентным оператором}.
\end{definition}

\begin{definition}[спектр]
	Множество $\sigma(A) = \mathbb{C} \setminus \rho(A)$ называется 
	\textit{спектром} оператора $A$.
\end{definition}

Рассмотрим $\lambda \in \sigma(A)$. Может быть два случая:
\begin{enumerate}
	\item
	$\Ker (\lambda I - A) \neq \left\{ \boldsymbol{0}\right\}$. Это значит, что оператор 
	$\lambda I - A$ имеет нетривиальное собственное подпространство, в котором (по определению)
	выполняется $Ax=\lambda x$, $x \neq \boldsymbol{0}$, для некоторых $x$
	(то, что часто называется собственными числами и векторами).
	\item
	$\Ker (\lambda I - A) = \left\{ \boldsymbol{0}\right\}$. 
	Здесь необходимо рассмотреть два подслучая:
	\begin{enumerate}
		\item
		$\dim X < +\infty$. В конечномерном случае из сюрьективности следует
		биективность, поэтому обратный оператор всегда существует.
		А спектр будет состоять из собственных значений.
		\item
		$\dim X = +\infty$. В этом случае может отсутствовать непрерывная обратимость.
		Если при этом $\Cl R(\lambda I - A) = X$, то говорят, что $\lambda$ 
		принадлежит непрерывной части спектра. Иначе говорят, что
		$\lambda$ принадлежит остаточной части спектра.
		(те $\lambda$ для которых ядро нетривиально называют дискретной частью спектра).
	\end{enumerate}
\end{enumerate}
