\section{Линейные операторы в банаховых пространствах}
\subsection{Сопряженный оператор}

Здесь и далее, если не оговорено иного, считаем, что мы находимся в B-пространствах.

\begin{definition}[сопряженное пространство]
    $X^* = \left\{ f : X \xrightarrow[\text{непр.}]{\text{лин.}} \mathbb{R} \right\}$ "--- пространство сопряженное к $X$.
\end{definition}
Заметим, что это пространство линейных функционалов, а значит, мы можем ввести в нем норму как норму линейного функционала.
\begin{gather}
    \norm{f} = \sup_{\mathclap{\norm{x} \leqslant 1}} \abs{f(x)}  \\
\end{gather}
По свойствам числовой оси получаем, что $X^*$ всегда банахово (независимо от $X$).

Рассмотрим теперь $A \in \mathcal{L}(X, Y)$. Пусть $f(x) = \phi(Ax)$, где $\phi \in Y^*$.
\begin{definition}
    Сопряженный оператор к $A$ имеет вид $A^*(\phi) = \phi \circ A$.
\end{definition}
\begin{statement}
    Если $A$ нерперывный, то $A^*$ тоже непрерывный.
\end{statement}
\begin{proof}
    Пусть $A$ непрерывен, тогда он ограничен. Тогда справедливо
    \begin{gather}
        \norm{A^*(\phi)} \leqslant \norm{\phi} \cdot \norm{A}\text{.} \label{1.2}
    \end{gather}
    Переходя к $\sup$ по $\phi$ получаем непрерывность $A^*$.
\end{proof}
\begin{theorem} \label{th1.2}
    $\norm{A^*} = \norm{A}$
\end{theorem}
\begin{proof}
    Мы доказали неравенство в одну сторону (неравенство \ref{1.2}). Докажем в другую.
    По определению $\sup$:
    $\forall \epsilon > 0, \exists x_{\epsilon} : \norm{x_{\epsilon}} = 1 \implies \norm{A} - \epsilon < \norm{Ax_{\epsilon}}$.
    Пусть $Z = \mathcal{L}(Ax_{\epsilon})$. Рассмотрим $f : Z \rightarrow \mathbb{R}$, $f(z) = \alpha\norm{Ax_{\epsilon}}$.
    Очевидно, что $f \in Y^*$.
    Поэтому, по теореме Хана-Банаха распространим $f$ на все $Y$, и назовем ее $\phi_{\epsilon}$.
    Тогда, по свойствам $f$, $\norm{\phi_{\epsilon}} = 1$, $\phi_{\epsilon}(Ax_{\epsilon}) = \norm{Ax_{\epsilon}}$.
    Слeдовательно, $\norm{A} - \epsilon < \phi_{\epsilon}(Ax_{\epsilon}) = A^*(\phi_{\epsilon}, x_{\epsilon})$.
    Тогда $\norm{A} - \epsilon < \norm{A^*} \cdot \norm{\phi_{\epsilon}} \cdot \norm{x_{\epsilon}} = \norm{A^*}$.
    Переходя к $\sup$ по $\epsilon$ получаем нужное неравенство.
\end{proof}

Пример: \todo

\begin{theorem}[теорема Рисса]
    Пусть $H$ "--- гильбертово пространство.
    Тогда $\forall f \in H^*$, $f$ можно представить как $f(x) = \scalarp{x}{y}$, где $y \in H$, $\norm{f} = \norm{y}$.
\end{theorem}
\begin{proof}
    Докажем в 3 этапа.
    \begin{enumerate}
        \item Построим соотвестсвующий функционал по данному $y$. \\
            Пусть $g(x) = \scalarp{x}{y}$. Очевидно, что это линейный функционал.
            По неравенству Шварца $\abs{g(x)} \leqslant \norm{y} \norm{x} \implies \norm{g} \leqslant \norm{y}$.
            Это значит, что $g$ ограничен.
            Возьмем $x = \frac{y}{\norm{y}}$.
            \[
                g\mathopen{}\left(\frac{y}{\norm{y}}\mathclose{}\right) = \scalarp{\frac{y}{\norm{y}}}{y}
                    = \frac{1}{\norm{y}}\scalarp{y}{y} = \norm{y}
            \]
            Сопоставляя это с тем, что $\norm{g} \leqslant \norm{y}$, получаем, что $\norm{g} = \norm{y}$.
        \item Докажем, что этому функционалу соответствует только один $y$. \\
            Пусть для какого-то $\widetilde{y}$ справедливо $g(x) = \scalarp{x}{\widetilde{y}}$.
            Тогда $0 = \scalarp{x}{y} - \scalarp{x}{\widetilde{y}} = \scalarp{x}{y - \widetilde{y}}$.
            Пусть $x = y - \widetilde{y}$, тогда
            $\scalarp{y - \widetilde{y}}{y - \widetilde{y}} = 0 \implies y = \widetilde{y}$
        \item Найдем $y$ для данного функционала $f$. \\
            Рассмотрим произвольный функционал $f \in H^*$.
            Как известно, $\Ker f$ "--- гиперплоскость, т.е.
            $\codim H_1 = \dim H_2 = 1$, где $H_1 = \Ker f$, $H_2 = H_1^\bot$, и $H = H_1 \oplus H_2$.
            Это по определению значит, что $x$ единственным образом представим как $x = x_1 + x_2$, где $x_1 \in H_1$, $x_2 \in H_2$.
            Поэтому, $f(x) = f(x_1) + f(x_2) = f(x_2) = f(\alpha e) = \alpha \cdot f(e)$,
            так как $x_1 \in \Ker f$, а $e$ "--- базисный вектор из $H_2$.
            Итак, $\alpha \cdot f(x) = \scalarp{x}{y} \Leftrightarrow f(e) = \scalarp{e}{y}$.
            Очевидно, $y$ можно брать из $H_2$, так как если у него будет компонента из $\Ker f$, то она будет ортогональна $e$.
            Поэтому, считаем, что $y = \beta e$.
            Получаем $f(e) = \scalarp{e}{\beta e} = \beta \cdot \norm{e}^2$.
            Положим $\beta = \frac{f(e)}{\norm{e}^2}$, тогда $y = \frac{f(e)}{\norm{e}^2}e$.
    \end{enumerate}
\end{proof}


Пример: \todo

Пусть $H = L_2(E)$, $\varphi \in L_2^*(E)$.
Тогда $\varphi(f) = \int_{E} g \cdot f d\mu$.
Согласно теореме Рисса, возвращаясь к сопряженному оператору, мы видим следующее.
$A^*(\varphi, x) = \varphi(Ax) = \scalarp{Ax}{y} = \scalarp{x}{z}$ - последнее равенство по теореме Рисса.
Причем $y$ и $z$ выбираются единственным образом, и $z = A^*(y)$.
В гильбертовом пространстве это может служить определением сопряженного оператора:
\begin{definition}[Сопряженный оператор в гильбертовом пространстве]
    Пусть $x, y \in H$. Пусть $A: H \rightarrow H$. Тогда $A^*$ - такой, что $\scalarp{Ax}{y} = \scalarp{x}{A^*y}$.
\end{definition}

\subsection{Ортогональное дополнение в банаховых пространствах}


\begin{definition}[ортогональное дополнение в B-пространстве]
    Пусть $S \subset X$.

    Тогда $S^\bot = \left\{ f \mid f \in X^*, \forall x \in S \implies f(x) = 0 \right\}$.
\end{definition}

\begin{definition}[ортогональное дополнение в сопряженном пространстве]
    Пусть $S \subset X^*$.

     Тогда $S^\bot = \left\{ x \mid x \in X, \forall f \in S \implies f(x) = 0 \right\}$.
\end{definition}
Заметим, что независимо от $S$, $S^\bot$ замкнуто в силу непрерывности $f(x)$

\begin{statement} \ 
    \begin{enumerate}
        \item $X^\bot = \left\{ 0\right\}$;
        \item $X^{*\bot} = \left\{ 0\right\}$.
    \end{enumerate}
\end{statement}
\begin{proof}\ 
    \begin{enumerate}
        \item $f \in X^\bot$. Если $\forall x \in X, f(x) = 0 $, то $f \equiv 0$.
        \item Рассмотрим $\forall f \in X^*$. Очевидно, $f(0) = 0$, а это значит, что $0 \in X^{*\bot}$.
            Предположим, что $\exists x_0 \neq 0 : x_0 \in X^{*\bot}$.
            По теореме \ref{th1.2}:
            \[
                \exists f \in X^* : f(x_0) = \norm{x_0} \neq 0 \implies x_0 \not\in X^{*\bot}\text{.} \qedhere
            \]
    \end{enumerate}
\end{proof}

\begin{definition}[множество значений оператора]
    $R(A) \defeq \left\{ Ax \mid x \in X\right\}$.
\end{definition}

\begin{theorem}
    $\Cl R(A) = \left( \Ker A^*\right)^\bot$.
\end{theorem}
\begin{proof}\ 
    \begin{enumerate}
        \item
            Пусть $y \in R(A)$, это значит, что $y = Ax$ для некоторого $x$.
            Рассмотрим $\phi \in \Ker A^*$. По определению, $A^*\phi = 0$, это значит,
            что $\forall x \in X \implies \phi(Ax) = \phi(y) = 0$.
            Следовательно, $y \in \left( \Ker A^*\right)^\bot$
        \item
            Пусть теперь $y \in \Cl R(A) \implies \exists y_n : y_n \rightarrow y$.
            По предыдущему пункту, $y_n \in \left( \Ker A^*\right)^\bot$.
            $\forall \phi \in \Ker A^* \implies \phi(y_n) = 0$, при этом, $\phi$ непрерывен.
            $\phi(y_n) \rightarrow \phi(y) = 0 \implies y \in \left( \Ker A^*\right)^\bot$.
        \item
            Осталось проверить, что $\left( \Ker A^*\right)^\bot \subset \Cl R(A)$.
            Вместо этого, мы проверим эквивалентный факт:
            $y \not\in \Cl R(a) \implies y \not\in \left( \Ker A^*\right)^\bot$.
            Итак, пусть $L = \Cl R(A)$. Очевидно, это линейное подпространство в $Y$.
            Пусть $\widehat{L} = \left\{ z + ty \mid z \in L, t \in \mathbb{R}\right\}$.
            Очевидно, $\widehat{L}$ "--- линейное подпространство $Y$.
            Рассмотрим $\phi : X \rightarrow \mathbb{R}$, $\phi(z + ty) \defeq t$.
            По теорема Хана-Банаха его можно продлить на $Y$ с сохранением нормы:
            $\exists \widehat{\phi} \in Y^* : \restr{\widehat{\phi}}{\widehat{L}} = \phi$.
            Причем, если $z \in L$, то $\widehat{\phi}(z) = 0$, значит $\widehat{\phi} \in \Ker A^*$.
            Но, при этом $\widehat{\phi}(y) = 1 \implies y \not\in \left( \Ker A^*\right)^\bot$. \qedhere
    \end{enumerate}
\end{proof}
\begin{theorem}
    $R(A) = \Cl R(A) \implies R(A^*) = \left( \Ker A\right)^\bot$.
\end{theorem}
\begin{proof}
    Рассмотрим $f \in R(A^*)$. По определению, для некоторого $\phi$, $f = A^*\phi$.
    Возьмем теперь $x \in \Ker A$. $Ax = 0 \implies f(x) = (\phi \circ A)(x) = \phi(Ax) = \phi(0) = 0$.
    Значит, $R(A^*) \subset \left( \Ker A\right)^\bot$.

    Пусть теперь $f \in \left( \Ker A\right)^\bot$. В силу того, что $R(A)$ - B-пространство
    (как замкнутое линейное подпространство другого B-пространства),
    Возьмем произвольный $y \in R(A)$, и $x$ такой, что $y = Ax$,
    и запишем $\phi$ как $\phi(y) \defeq f(x)$. Покажем, что такое определение
    действительно корректное.
    Пусть $y = Ax'$; тогда $A(x - x') = 0 \implies x - x' \in \Ker A$.
    Поэтому $f(x - x') = 0 \implies f(x) = f(x')$. Это значит, что значение $\phi$ не
    зависит от выбора конкретного $x$. Значит, наша формула корректная.
    Осталось показать ограниченность $\norm{\phi}$.

    Рассмотрим ассоциированный оператор
    $\mathcal{U}_A : X/\Ker A \rightarrow R(A)$.
    Покажем, что он непрерывен.

    $\norm{\mathcal{U}_A} = \sup_{\norm{\eqcls{x}} = 1} \norm{\mathcal{U}_A\eqcls{x}}$,
    так как $\norm{\eqcls{x}} = \inf_{z \in \eqcls{x}} \norm{z} = 1$,
    то существует $x' \in \eqcls{x} : \norm{x'} \leqslant 2$.
    Возьмем $x'$ в качестве представителя. Тогда
    \begin{equation}
        \begin{split}
            \norm{\mathcal{U}_A} & =
            \sup_{\mathclap{\norm{\eqcls{x}} = 1}} \norm{\mathcal{U}_A\eqcls{x}}
            \leqslant \sup_{\mathclap{\norm{x} \leqslant 2}} \norm{Ax}
            \leqslant \sup_{\mathclap{\norm{y} \leqslant 1}} \norm{A(2y)} \\
            & = 2 \cdot \sup_{\mathclap{\norm{y} \leqslant 1}} \norm{Ay}
            = 2 \norm{A}
        \end{split}
    \end{equation}
    Заметим еще, что он биективен, так как все точки $x$ для которых $y = Ax$
    (для какого-то одного фиксированного $y$) лежат в одном классе эквивалентности.
    Это значит, что по теореме Банаха о гомеоморфизме, $\mathcal{U}_A^{-1}$ непрерывен.
    Напомним, что норма на элементах $X/\Ker A$ определяется как
    \begin{gather}
        \norm{\eqcls{x}} \defeq \inf_{\mathclap{z \in \eqcls{x}}} \norm{z}
    \end{gather}
    По непрерывности обратного оператора получаем $\norm{\eqcls{x}} \leqslant K \cdot \norm{y}$.
    Нам нужно сделать неравенство строгим, поэтому считаем, что
    $\norm{\eqcls{x}} < 2K \cdot \norm{y}$. Дальше, по определению инфимума,
    $\exists z \in \eqcls{x} : \norm{z} < 2K \cdot \norm{y}$. Значит, $z - x \in \Ker A$.
    В силу того, что значение функционала $f$ одно и то же внутри класса эквивалентности,
    можно вместо $x$ взять $z$. Таким образом,
    $\abs{\phi(y)} \leqslant 2K \cdot \norm{f} \cdot \norm{y}$, из этого следует, что
    $\phi$ непрерывен.
    Далее, по теореме Хана-Банаха, продолжим $\phi$ на все пространство и получим, что
    $\exists \widehat{\phi} \in Y^* : f = A^*\widehat{\phi} \implies f \in R(A^*)$.
\end{proof}

В силу того, что во второй теореме требуется замкнутость,
возникает вопрос: а когда это действительно будет?
Одним из инструментов, дающих ответ на этот вопрос, является
\emph{априорная оценка решения операторного уравнения}.

\begin{definition}[априорная оценка решения операторного уравнения]
    Пусть $A : X \rightarrow Y$ - линейный оператор,
    $y \in R(A)$, $\exists \alpha=\text{const}$,
    такая что $\norm{x} \leqslant \alpha \norm{y}$,
    где $y = Ax$.
    Коэффициент $\alpha$ называется \emph{априорной оценкой}.
\end{definition}

Ответ на поставленный вопрос дает следующая теорема:
\begin{theorem}
    Если $A$ "--- линейный ограниченный оператор, такой что для уравнения
    $y = Ax$ существует априорная оценка, то $R(A)$ "--- замкнуто.
\end{theorem}

\begin{proof}
    Рассмотрим последовательность значений оператора $y_n \in R(A)$,
    такую что $y_n \rightarrow y$. Проверим, что тогда $y \in R(A)$.
    Пусть $\epsilon_n = \frac{1}{2^n}$, в силу банаховости пространства $Y$,
    можем написать ряд утверждений:
    \begin{gather*}
        \text{для } \epsilon_1 \: \exists n_1 : \forall n, m \geqslant n_1
        \implies \norm{y_m - y_n} \leqslant \epsilon_1 \\
        \text{для } \epsilon_2 \: \exists n_2 : \forall n, m \geqslant n_2
        \implies \norm{y_m - y_n} \leqslant \epsilon_2 \\
        \dots \\
        \text{для } \epsilon_k \: \exists n_k : \forall n, m \geqslant n_k
        \implies \norm{y_m - y_n} \leqslant \epsilon_k \\
        \dots \\
    \end{gather*}
    при этом, очевидно, что $n_k \leqslant n_{k + 1}$.
    Теперь рассмотрим ряд
    $y_{n_1} + (y_{n_2} - y_{n_1}) + (y_{n_3} - y_{n_2}) + \dotso = y$.
    Слагаемое этого ряда мажорируется сходящейся геометрической прогрессией,
    поэтому, он сходится абсолютно.

    Так как $R(A)$ "--- подпространство, значит $y_{n_{k + 1}} - y_{n_k} \in R(A)$.
    Следовательно, $y_{n_{k + 1}} - y_{n_k} = Ax_k$. По условию теоремы, для $x_k$
    выполняется $\norm{x_k} \leqslant \alpha \norm{y_{n_{k + 1}} - y_{n_k}}
    \leqslant \alpha \epsilon_k$. Возьмем ряд $x_0 + x_1 + x_2 + \dots$,
    где $y_{n_1} = Ax_0$.
    Ряд из норм его слагаемых можно ограничить сходящимся рядом:
    $\norm{x_0} + \norm{x_1} + \norm{x_2} + \dotso \leqslant
     \norm{x_0} + \alpha \cdot (\epsilon_1 + \epsilon_2 + \dotso) =
     \norm{x_0} + \alpha$.
    Поэтому у него есть предел $x$, и мы можем применить к нему оператор почленно
    (в силу его непрерывности):
    $Ax = Ax_0 + Ax_1 + Ax_2 + \dotso = y_{n_1} + (y_{n_{2}} - y_{n_1}) +
    (y_{n_{3}} - y_{n_2}) + \dotso = y$. Таким образом, $y \in R(A)$.
\end{proof}

\pagebreak
