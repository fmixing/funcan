\documentclass{article}

\usepackage{enumitem,linegoal}
\usepackage{xfrac}
\usepackage{faktor}
\usepackage{stmaryrd}
\usepackage{calc}
\usepackage{wrapfig}
\usepackage{blindtext}
\usepackage{tikz}
\usepackage{yhmath}
\usepackage{amssymb}
\usepackage{amsthm}
\usepackage{amsmath}
\usepackage{mathtext}
\usepackage[T1,T2A]{fontenc}
\usepackage[utf8]{inputenc}
\usepackage[russian]{babel}
\usepackage[left=2cm,right=2cm,top=2cm,bottom=2cm,bindingoffset=0cm]{geometry}
\usepackage[mathscr]{euscript}
\usepackage{microtype}
\usepackage{enumitem}
\usepackage{bm}
\usepackage{listings}
\usepackage{cancel}
\usepackage{proof}
\usepackage{epigraph}
\usepackage{titlesec}
\usepackage{mathtools}
\usepackage[hidelinks]{hyperref}

\selectlanguage{russian}

\hypersetup{%
    colorlinks=true,
    linkcolor=blue
}

\DeclareMathOperator{\grad}{grad}
\DeclareMathOperator{\Div}{div}
\DeclareMathOperator{\rot}{rot}
\DeclareMathOperator{\im}{Im}
\DeclareMathOperator{\Ker}{Ker}
\DeclareMathOperator{\codim}{codim}
\DeclareMathOperator{\Cl}{Cl}
\DeclareMathOperator{\Int}{Int}

\title{%
    \sc{Функциональный анализ, 6 семестр} \\
    \large Конспект лекций Додонова Н. Ю.}
\author{\url{https://github.com/artemZholus/funcan}}
\date{}

\begin{document}

\theoremstyle{definition}
\newtheorem*{definition}{Определение}
\theoremstyle{plain}
\newtheorem{theorem}{Теорема}[section]
\newtheorem{axiom}{Аксиома}
\newtheorem{lemma}[theorem]{Лемма}
\newtheorem{statement}[theorem]{Утверждение}
\newtheorem{nb}[theorem]{N. B.}
\newtheorem{corollary}[theorem]{Следствие}
\theoremstyle{remark}
\newtheorem*{example}{Пример}
\newtheorem{property}[theorem]{Свойство}

\newcommand{\todo}{\textsc{\textbf{TODO}}}
\newcommand{\abs}[1]{\left|#1\right|}
\newcommand{\eqcls}[1]{\left[#1 \right]}
\newcommand{\norm}[1]{\left\|#1\right\|}
\newcommand{\bigslant}[2]{{\raisebox{.2em}{$#1$}\left/\raisebox{-.2em}{$#2$}\right.}}
\newcommand{\normp}[1]{\norm{#1}_p}
\newcommand\restr[2]{{\left.\kern-\nulldelimiterspace #1 \vphantom{\big|} \right|_{#2} }}
\newcommand{\normpp}[2]{\norm{#1}_{#2}}
\newcommand{\defeq}{\mathrel{\stackrel{\makebox[0pt]{\mbox{\normalfont\tiny def}}}{=}}}
\makeatletter
\newcommand*{\rom}[1]{\expandafter\@slowromancap\romannumeral #1@}
\makeatother
\newcommand{\pdiff}[2]{\frac{\partial #1}{\partial #2}}
\newcommand{\feps}{\forall\varepsilon}
\newcommand{\scalarp}[2]{\left\langle #1 , #2\right\rangle}
\newcommand{\set}[1]{\left\{#1\right\}}
\renewcommand{\phi}{\varphi}
\renewcommand{\epsilon}{\varepsilon}


\begin{theorem}
    $A$ "--- компактный оператор, тогда $\sigma(A)$ не
    более чем счетно и может состоять только из 0.
\end{theorem}
\begin{proof}
    Рассмотрим $\alpha > 0$. Покажем, что в отрезке $[\alpha,  \norm{A}]$ лежит конечное число точек спектра $\lambda$.
    
    По модулю все точки $\abs{\lambda}$ находятся в $[0, \norm{A}]$.
    

    \begin{tikzpicture}
    \draw  (0,0) node[circle,fill,inner
        sep=1pt,label=above:$0$](a){} 
        -- (1,0) node[circle,fill,inner sep=1pt,label=above:$\alpha$](b){} 
        -- (2.5,0) node[below] {конечное} 
        -- (4,0) node[circle,fill,inner sep=1pt,label=above:$\norm{A}$](c){} ;
    \end{tikzpicture}

    Будем доказывать от противного. Пусть таких точек бесконечно много, тогда выделим подпоследовательность таких точек $\lambda_{n}$, что $\lambda_{n} \neq \lambda_{m}$ при $n \neq m$, $\alpha < \abs{\lambda_{n}}$, $\lambda_{n} \in A$, $n \in \mathbb{N}$ $\lambda_{n}$ --- собственное число.
    
    Возьмем $x_{1} \neq 0$, $Ax_{1} = \lambda_{1}x_{1}$, $x_{2} \neq 0$, $Ax_{2} = \lambda_{2}x_{2}$. 
    Если допускать, что $x_{2} = \gamma x_{1}$, то $Ax_{2} = \gamma Ax_{1}$, следовательно $\lambda_{2}x_{2} = \gamma \lambda_{1}x_{1}$, и значит $x_{2} = \frac{\lambda_{1}}{\lambda_{2}} \gamma x_{1}$, но $x_{2} = \gamma x_{1}$, и если $\gamma \neq 0$, то $\gamma = \frac{\lambda_{1}}{\lambda_{2}} \gamma \implies \lambda_{1} = \lambda_{2}$.
    
    Таким образом, $\{x_{1}, x_{2}\}$ --- линейно независимая пара точек.
    
    Двигаемся так дальше по индукции, на n-м шаге $\{x_{1}, \dotso, x_{n}\}$ --- линейно независимые, $\lambda_{n}x_{n} = A x_{n}$. 
    
    $\exists x_{n+1} \neq 0: \lambda_{n+1}x_{n+1} = Ax_{n+1}$. Тогда $x_{n+1} = \sum_{1}^{n}\gamma_k x_k$ --- единственно. Применим оператор $A$ к этому равенству. $Ax_{n+1} = \sum_{1}^{n}\gamma_k Ax_k$, $\lambda_{n+1} x_{n+1} = \sum_{1}^{n}\gamma_k \lambda_k x_k$, $x_{n+1} = \sum_{1}^{n}\gamma_k \frac{\lambda_k}{\lambda_{n+1}} x_k$ --- единственно, так как $\{x_{i}\}$ --- линейно независимые, то хотя бы одна $\lambda_{k_{0}} \neq 0$. Приравниваем две суммы и получаем $\gamma_{k_{0}} \frac{\lambda_{k_{0}}}{\lambda_{n+1}} = \gamma_{k_{0}}$, и значит $\lambda_{k_{0}} = \lambda_{n+1}$.


   

   

\end{proof}



\end{document}
